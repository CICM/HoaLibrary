Sound space is one of the first dimensions of the contemporary musical thought, specialy in the electroacoustic music domain but also in intermedia arts. In this context, the C\-I\-C\-M has made of the spatialization its main research axis. This projects « La spatialisation du son par les musiciens pour les musiciens » and « Des Interfaces pour la mise en espace du son » developed as a part of the L\-A\-B\-E\-X Arts H2\-H aims to give spatialization models based on high order ambisonics to musicians and graphical interfaces to facilitate the manipulation of signal processing. It takes place between the M\-S\-H Paris-\/\-Nord and the plate-\/forme Arts-\/\-Sciences-\/\-Technologies. Partners are Louis-\/\-Lumière, G\-R\-A\-M\-E, the Muse en Circuit, the University of Montréal and the Z\-K\-M. The I\-R\-C\-A\-M and the G\-M\-E\-A have also participated to the project. 